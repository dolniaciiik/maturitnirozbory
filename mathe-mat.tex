\documentclass[12pt, a4paper]{article}
\usepackage[czech]{babel}
\usepackage{amsmath, amssymb}
\usepackage{tikz}
\usepackage{float}
\floatstyle{boxed} 
\restylefloat{figure}
\newcommand{\imply}{\Rightarrow}
\newcommand{\equiva}{\Leftrightarrow}

\title{Matematika pro dementy}
\author{Fantomas}
\date{Únor 2025}

\begin{document}
\maketitle
\pagebreak
\begin{abstract}
Vypracované otázky z matematiky, tipy a triky a tak
\end{abstract}
\pagebreak

\tableofcontents

\pagebreak
\section{Výroková logika a množiny}

\subsection*{Množiny}
Množinou rozumíme souhrn nějakých objektů (prvků). Zápis $x \in M$ znamená že prvek \textit{x} náleží množině \textit{M}. Množinu můžeme určit výčtem prvků, charakteristickou vlastností nebo 
množinovými operacemi. Rovnost množin znamená, že každý prvek množiny \textit{M} je prvkem množiny \textit{N} a současně každý prvek množiny \textit{N} je prvkem množiny \textit{M}.

\subsubsection*{Podmnožina}
Množinu \textit{M} nazýváme podmnožinou množiny \textit{N}, právě když je každý prvek množiny \textit{M} prvkem množiny \textit{N}. Zápis symbolem $\subseteq$ nebo $\subset$; $M\subset N$ značí, že \textit{M} je vlastní podmnožinou množiny \textit{N}, tedy $M \neq N$; $M \subseteq N$ značí nevlastní podmnožinu, tedy $M \subset N$ nebo $M = N$.

\subsubsection*{Charakteristická vlastnost}
Zápis $A=\{x \in M; vlastnost\}$, kde každý prvek z množiny \textit{M}, mající danou vlastnost, patří do množiny \textit{A}.

\subsection*{Množinové operace}
\textbf{Sjednocení} $A \cup B$, je množina všech prvků, patřících alespoň do jedné z množin \textit{A}, \textit{B}.\\
\textbf{Průnik} $A \cap B$, je množina všech prvků, patřících zároveň do obou množin \textit{A}, \textit{B}.\\
\textbf{Rozdíl} $A \setminus B$, je množina všech prvků, patřících do množiny \textit{A} a \textbf{nepatřících} do množiny \textit{B}.\\
! Sjednocení i průnik jsou komutativní a asociativní operace.\\
\textbf{Doplněk} $A'_M $ množiny A v množině M je množina všech prvků množiny M, které nepatří do množiny A $\Rightarrow A'_M = M \setminus A$.

\subsection*{Intervaly}
Nechť \textit{a}, \textit{b} jsou dvě reálná čísla, že $a<b$, pak\\
$(a,b) = \{x \in \mathbb{R}; a < x < b\}$ je otevřený interval\\
$(a,b\rangle = \{x \in \mathbb{R}; a < x \leq b\}$ je polootevřený interval\\
$\langle a,b) = \{x \in \mathbb{R}; a \leq x < b\}$ je polouzavřený interval\\
$\langle a,b \rangle = \{x \in \mathbb{R}; a \leq x \leq b\}$ je uzavřený interval\\

\subsection*{Výroky}
Výrokem rozumíme sdělení, o kterém má smysl uvažovat jeho pravdivost. Každý výrok má \textbf{pravdivostní hodnotu}, 0 (nepravda) nebo 1 (pravda).\\
\textbf{Hypotéza} je výrok jehož pravdivostní hodnotu neznáme. \\
\textbf{Výroková formule} je tvrzení s proměnou, po dosazení se stane výrokem.

\subsubsection*{Negace výroku}
\textbf{Negace výroku}, \textit{"Není pravda, že A"}, zapisujeme $\neg A $, vždy opačná pravdivostní hodnota.

\subsubsection*{Logické operátory}
Pomocí těchto operátorů tvoříme \textbf{složené výroky} nebo \textbf{formule}.\\
\textbf{Konjunkce}, "A a současně (et) B", zapisujeme $A \land B$
\begin{center}
\begin{tabular}{|c | c | c|} 
\hline
A & B & $A \land B$ \ \\
\hline
0 & 0 & 0 \\
\hline
0 & 1 & 0 \\
\hline
1 & 0 & 0 \\
\hline
1 & 1 & 1\\
\hline
\end{tabular}
\end{center}
\textbf{Disjunkce}, "A nebo (vel) B", zapisujeme $A \lor B$
\begin{center}
\begin{tabular}{|c | c | c|} 
\hline
A & B & $A \lor B$ \ \\
\hline
0 & 0 & 0 \\
\hline
0 & 1 & 1 \\
\hline
1 & 0 & 1 \\
\hline
1 & 1 & 1\\
\hline
\end{tabular}
\end{center}
\textbf{Ostrá disjunkce}, "Buď A, nebo B", zapisujeme $ A \veebar B $
\begin{center}
\begin{tabular}{|c | c | c|} 
\hline
A & B & $A \veebar B$ \ \\
\hline
0 & 0 & 0 \\
\hline
0 & 1 & 1 \\
\hline
1 & 0 & 1 \\
\hline
1 & 1 & 0\\
\hline
\end{tabular}
\end{center}
\textbf{Implikace}, "Z A plyne B", zapisujeme $A \Rightarrow B$
\begin{center}
\begin{tabular}{|c | c | c|} 
\hline
A & B & $A \Rightarrow B$ \ \\
\hline
0 & 0 & 1 \\
\hline
0 & 1 & 1 \\
\hline
1 & 0 & 0 \\
\hline
1 & 1 & 1\\
\hline
\end{tabular}
\end{center}
\textbf{Ekvivalence}, "A je ekvivalentní s B.", "A právě tehdy, když B.", zapisujeme $A \Leftrightarrow B$ 
\begin{center}
\begin{tabular}{|c | c | c|} 
\hline
A & B & $A \Leftrightarrow B$ \ \\
\hline
0 & 0 & 1 \\
\hline
0 & 1 & 0 \\
\hline
1 & 0 & 0 \\
\hline
1 & 1 & 1\\
\hline
\end{tabular}
\end{center}

\subsubsection*{Tautologie}
\textbf{Tautologie} je výrok/formule, který je vždy pravdivý.\\ 
\textbf{Kontradikce} je výrok/formule, který je vždy nepravdivý.\\
\textbf{Důležité tautologie}:
\begin{itemize}
\item $\neg(\neg A) \equiv A$ 
\item $\neg (A \Rightarrow B) \equiv A \land \neg B$ 
\item $ A \Rightarrow B \equiv \neg B \Rightarrow \neg A $
\item $ \neg (A \land B) \equiv \neg A \lor \neg B$
\item $ A \Rightarrow B \equiv \neg A \lor B $
\item $ \neg (A \Leftrightarrow B) \equiv A \veebar B $
\item $ \neg (A \lor B) \equiv \neg A \land \neg B $
\item $ A \Leftrightarrow B \equiv (A \Rightarrow B) \land (B \Rightarrow A) $
\item $ \neg (A \veebar B) \equiv A \Leftrightarrow B$
\end{itemize}
\pagebreak

\subsection*{Kvantifikace výrokových formulí}
Výroková formule $\varphi (x)$, obsahující proměnnou x, se stane výrokem po kvantifikaci \textit{x}.\\
\textbf{Obecný kvantifikátor} $\forall$, "pro každé, pro všechna, ..." \\
\textbf{Malý kvantifikátor} $\exists$, "existuje alespoň jedno, nějaké, ..." \\
Př.: Formuli $\varphi (x) \sim x > 0$ lze kvantifikovat:\\
$(\forall x \in \mathbb{N}) x > 0$ ... Všechna přirozená čísla jsou kladná.\\
$ (\exists x \in \mathbb{N}) x > 0$ ... Existuje alespoň jedno přirozené číslo větší než 0.\\

\subsubsection*{Negace kvantifikátorů}
Negace výroku $(\forall x) \varphi (x) $ je výrok $(\exists x) \neg \varphi (x)$.\\
Negace výroku $(\exists x) \varphi (x) $ je výrok $ (\forall x) \neg \varphi (x)$.\\

\subsection*{Věta, definice, důkaz, správné úsudky}
\textbf{Matematická věta} je důležité, netriviální a dostatečně obecné tvrzení neboli výrok. Věta obsahuje předpoklad a závěr.
\textbf{Axiom (postulát)} je tvrzení, které se předem předpokládá za platné. 
\textbf{Definice} slouží k zavedení nových pojmů; stanoví nový pojem a určí ho pomocí již stanovených.
\subsubsection*{Správný úsudek}
\textbf{Správný úsudek} je takový, kdy je z pravdivých premis vyvozen pravdivý závěr.\\
\textbf{Zákon vyloučení možnosti}:
\begin{gather*}
 p \lor q \\ \neg p \\ \hline q
\end{gather*}
\textbf{Zákon odloučení}:
\begin{gather*}
p \Rightarrow q \\ p \\ \hline q
\end{gather*}
\textbf{Zákon nepřímé úvahy}:
\begin{gather*}
p \Rightarrow q \\ \neg q \\ \hline \neg p
\end{gather*}
\textbf{Zákon kontrapozice}:
\begin{gather*}
p \Rightarrow q \\ \hline \neg q \Rightarrow \neg p
\end{gather*}
\pagebreak

\section{Mnohočleny, mocniny a odmocniny}

Zápis $ 1 + \sqrt{1,5625-(\frac{3}{4})^2} $ je \textbf{číselný výraz} s hodnotou 2.\\
Zápis $ x^2 + 2xy +1 $ je výraz s proměnnými \textit{x}, \textit{y}.\\
\textbf{Definiční obor výrazu} je množina všech přípustných hodnot proměnné, pro které má výraz smysl.\\
Výraz $V = x^2 +1 $ má definiční obor $\mathbb{R}$\\
Výraz $V = \frac{1}{y}$ má smysl pro nenulové hodnoty \textit{y} \\
 $\Rightarrow D_V : y \in \mathbb{R} \setminus \{0\}$  nebo $ D_V = (-\infty,0) \cup (0,+\infty)$\\

\subsection*{Mnohočleny}
\textbf{Mnohočlen} (polynom) s jednou proměnnou je výraz $ a_nx^n + a_{n-1}x^{n-1} +...+ a_1x^1 + a_0 $, kde \textit{n} je stupeň mnohočlenu.\\
$a_1x + a_0$, resp. $ax+b$ je linearní dvojčlen.\\
$a_2x^2 + a_1x + a_0$, resp. $ax^2 + bx + c$ je kvadratický trojčlen.\\

\subsubsection*{Dělení mnohočlenu mnohočlenem}
$(4x^3 + 3x^2 - 2x -5):(x-1) = 4x^2 + 7x + 5$\\
$ 4x^3 - 4x^2$\\
\rule{2cm}{0.4pt}\\
$ 0x^3 + 7x^2 -2x$\\
$ 0x^3 + 7x^2 -7x$\\
\rule{3cm}{0.4pt}\\
$0x^3 + 0x^2 + 5x - 5$\\
$0x^3 + 0x^2 + 5x - 5$\\
\rule{4cm}{0.4pt}\\
$0x^3 + 0x^2 + 0x - 0$\\
pozn. zbytek stejně jako u číselného dělení\\

\subsubsection*{Umocňování}
\begin{itemize}
\item $(AB)^n = A^nB^n$
\item $(A+B)^2 = A^2 + 2AB + B^2$
\item $(A-B)^2 = A^2-2AB+B^2$
\item $(A+B)^3=A^3 + 3A^2B + 3AB^2 + B^3$
\item $(A-B)^3=A^3 - 3A^2B + 3AB^2 - B^3$
\item $A^2-B^2 =(A-B)(A+B)$
\item $A^3-B^3 = (A-B)(A^2+AB+B^2)$
\item $A^3+B^3 = (A+B)(A^2-AB+B^2)$
\item $(-A+B)^2 = (A-B)^2$
\item $(-A-B)^2 = (A+B)^2$
\item $(A+B)^n=\sum \binom{n}{k} a^{n-k}b^k $
\end{itemize}

\section{Lomené výrazy}
\subsection*{Rozšiřování a krácení lomených výrazů}
\textbf{Rozšířit lomený výraz} znamená vynásobit čitatele i jmenovatele stejným číslem.\\
\begin{center}
$\frac{x}{x+2}=\frac{x(x-2)}{(x+2)(x-2)}=\frac{x^2-2x}{x^2 - 4}$\\
\end{center}
\textbf{Krátit lomený výraz} znamená vydělit čitatele i jmenovatele stejným číslem.\\
\begin{center}
$\frac{a^2bc^3}{abc^2}=\frac{a^2bc^3:abc}{abc^2:abc}=\frac{ac^2}{c}=ac$\\
\end{center}

\subsection*{Sčítání a odčítání lomených výrazů}
Nejdříve rozložíme všechny jmenovatele na součin, určíme společný jmenovatel jako NSN všech jmenovatelů, každý LV rozšíříme na společný jmenovatel, sečteme a odečteme čitatele, rozložíme čitatele na součin a zkrátíme (je-li to možné) a určíme podmínky.\\
\[
\begin{aligned}
	V &= \frac{3+2x}{2-x}-\frac{2-3x}{2+x}+\frac{x(16-x)}{x^2-4}\\
	V &= \frac{3+2x}{2-x}-\frac{2-3x}{2+x}+\frac{x(16-x)}{(x+2)(x-2)}\\
	V &=  -\frac{(3+2x)(x+2)}{(x-2)(x+2)}-\frac{(2-3x)(x-2)}{(x+2)(x-2)}+\frac{x(16-x)}{(x+2)(x-2)}\\
	V &= \frac{-7x-6-2x^2-8x+4+3x^2+16x-x^2}{(x-2)(x+2)}\\
	V &= \frac{x-2}{(x-2)(x+2)}\\
	V &= \frac{1}{x+2}\\
\end{aligned}
\]

\subsection*{Vyjadřování neznámé ze vzorce}
Při vyjadřování neznámé ze vzorce využíváme:
\begin{itemize}
	\item záměna stran vzorce
	\item vynásobení/vydělení vzorce nenulovým číslem nebo výrazem
	\item přičtení/odečtení libovolného čísla nebo výrazu
	\item pokud jsou ve vzorci nezáporné veličiny, pak umocnění nebo odmocnění
\end{itemize} 

\subsection*{Výrazy s mocninami a odmocninami}
Pro každá reálná \textit{a}, \textit{b} a pro každá reálná \textit{r}, \textit{s} platí:
\begin{itemize}
	\item $a^r \cdot a^s = a^{r+s}$
	\item $ (a^r)^s = a^{r \cdot s} $
	\item $ \frac{a^r}{a^s} = a^{r-s} $
	\item $ (ab)^r = a^rb^r $
	\item $ (\frac{a}{b})^r = \frac{a^r}{b^r} $
	\item $ a^0 = 1, a \neq 0 $
	\item $ a^{-r} = \frac{1}{a^r} $
	\item $ \sqrt [s]{a^r} = a^{\frac{r}{s}}$
\end{itemize}

\section{Lineární rovnice a nerovnice}

\subsection*{Lineární rovnice}
\textbf{Lineární rovnice} má tvar $ax+b=0, a \neq 0$. Má jediný kořen $x=-\frac{b}{a}$.\\
Pokud užitím ekvivalentních úprav získáme tvar $0x+b=0$, pak má rovnice nekonečně mnoho řešení ($b=0$), nebo nemá řešení ($b \neq 0$).\\
\textbf{Definiční obor rovnice} je množina všech přípustných hodnot jejích kořenů; $x_1 \notin D_r \imply x_1 \notin K$.\\

\subsection*{Lineární nerovnice}
\textbf{Lineární nerovnice} má tvar: 
\begin{itemize}
\item $ax + b < 0$
\item $ax + b > 0$
\item $ax + b \leq 0$
\item $ax + b \geq 0$
\end{itemize}
Pokud lze nerovnici převést na tvar $ax + b \leq 0$:
\begin{center}
$b \leq 0 \imply K = \mathbb{R} \lor b > 0 \imply K = \{\}$
\end{center}
Pokud lze nerovnici převést na tvar $ax + b \geq 0$:
\begin{center}
$b \geq 0 \imply K = \mathbb{R} \lor b < 0 \imply K = \{\}$
\end{center}
Pokud lze nerovnici převést na tvar $ax + b < 0$:
\begin{center}
$b < 0 \imply K = \mathbb{R} \lor b \geq 0 \imply K = \{\}$
\end{center}
Pokud lze nerovnici převést na tvar $ax + b > 0$:
\begin{center}
$b > 0 \imply K = \mathbb{R} \lor b \leq 0 \imply K = \{\}$
\end{center}
\pagebreak

\subsection*{Grafické řešení lineární rovnice a nerovnice}
\textbf{Lineární funkce} je funkce s předpisem $y=ax+b$\\
Rovnici převedeme na tvar $ax+b=cx+d$ a budeme uvažovat $f(x)=ax+b$ a $g(x)=cx+d$, kořen leží v $f(x) \cap g(x)$.\\
\begin{figure}[H]
\centering
\begin{tikzpicture}
%% init-xy
\draw[help lines, color=gray!30, dashed] (-3.5,-3.5) grid (3.5,3.5);
\draw[->,ultra thick] (-3.5,0)--(3.5,0) node[right]{$x$};
\draw[->,ultra thick] (0,-3.5)--(0,3.5) node[above]{$y$};
%% init-xy
\draw [blue] (-1.5,3.5) -- (3.5,-1.5);
\draw [blue] (-1.25,-3.5) -- (2.25,3.5);
\filldraw [red] (1,1) circle (2pt);
\end{tikzpicture}
\caption{2x-1=2-x}
\end{figure}

\textbf{Lineární nerovnici} řešíme podobně jako rovnici: převedeme na tvar $ax+b=cx+d$ a budeme uvažovat $f(x)=ax+b$ a $g(x)=cx+d$, nerovnice mohou mít jeden z tvarů:\\
\begin{center}
\begin{tabular}{c  c  c  c} 
$-x-1>2x-\frac{5}{2}$ & $-x-1<2x-\frac{5}{2}$ & $-x-1 \geq 2x-\frac{5}{2}$ & $-x-1 \leq 2x-\frac{5}{2}$\\
$f(x)>g(x)$ & $f(x)<g(x)$ & $f(x) \geq g(x)$ & $f(x) \leq g(x)$\\
$K = (-\infty, \frac{1}{2})$ & $K = (\frac{1}{2}, \infty)$ & $K = (-\infty, \frac{1}{2} \rangle $ & $K = \langle \frac{1}{2}, \infty) $\\
 
ad 2 &
ad 3 &
ad 4 &
ad 5 \\
\end{tabular}
\end{center}

% ad 2
\begin{figure}[H]
\centering
\begin{tikzpicture}
% init-xy
\draw[help lines, color=gray!30, dashed] (-3.5,-3.5) grid (3.5,3.5);
\draw[->,ultra thick] (-3.5,0)--(3.5,0) node[right]{$x$};
\draw[->,ultra thick] (0,-3.5)--(0,3.5) node[above]{$y$};
% draw funcs
\draw [blue] (-3.5,2.5) -- (2.5,-3.5);
\draw [blue] (-0.5,-3.5) -- (3,3.5);
% draw roots
\draw [red] (-3.5,0) -- (0.5,0);
\draw [red] (0.5,0) circle (3pt);
\draw [gray, densely dotted] (0.5,-1.5) -- (0.5,0);
\end{tikzpicture}
\caption{$-x-1>2x-\frac{5}{2}$}
\end{figure}

% ad 3
\begin{figure}[H]
\centering
\begin{tikzpicture}
%% init-xy
\draw[help lines, color=gray!30, dashed] (-3.5,-3.5) grid (3.5,3.5);
\draw[->,ultra thick] (-3.5,0)--(3.5,0) node[right]{$x$};
\draw[->,ultra thick] (0,-3.5)--(0,3.5) node[above]{$y$};
%% init-xy
\draw [blue] (-3.5,2.5) -- (2.5,-3.5);
\draw [blue] (-0.5,-3.5) -- (3,3.5);
%draw roots
\draw [red]  (0.5,0) -- (3.5,0);
\draw [red] (0.5,0) circle (3pt);
\draw [gray, densely dotted] (0.5,-1.5) -- (0.5,0);
\end{tikzpicture}
\caption{ $-x-1<2x-\frac{5}{2}$ }
\end{figure}

% ad 4
\begin{figure}[H]
\centering
\begin{tikzpicture}
%% init-xy
\draw[help lines, color=gray!30, dashed] (-3.5,-3.5) grid (3.5,3.5);
\draw[->,ultra thick] (-3.5,0)--(3.5,0) node[right]{$x$};
\draw[->,ultra thick] (0,-3.5)--(0,3.5) node[above]{$y$};
%% init-xy
\draw [blue] (-3.5,2.5) -- (2.5,-3.5);
\draw [blue] (-0.5,-3.5) -- (3,3.5);
%draw roots
\draw [red] (-3.5,0) -- (0.5,0);
\filldraw [red] (0.5,0) circle (3pt);
\draw [gray, densely dotted] (0.5,-1.5) -- (0.5,0);
\end{tikzpicture}
\caption{$-x-1 \geq 2x-\frac{5}{2}$}
\end{figure}

% ad 5
\begin{figure}[H]
\centering
\begin{tikzpicture}
%% init-xy
\draw[help lines, color=gray!30, dashed] (-3.5,-3.5) grid (3.5,3.5);
\draw[->,ultra thick] (-3.5,0)--(3.5,0) node[right]{$x$};
\draw[->,ultra thick] (0,-3.5)--(0,3.5) node[above]{$y$};
%% init-xy
\draw [blue] (-3.5,2.5) -- (2.5,-3.5);
\draw [blue] (-0.5,-3.5) -- (3,3.5);
%draw roots
\draw [red]  (0.5,0) -- (3.5,0);
\filldraw [red] (0.5,0) circle (3pt);
\draw [gray, densely dotted] (0.5,-1.5) -- (0.5,0);
\end{tikzpicture}
\caption{$-x-1 \leq 2x-\frac{5}{2}$}
\end{figure}

\pagebreak

\subsection*{Rovnice a nerovnice v součinovém tvaru}
Při řešení využíváme $ab = 0 \equiva a = 0 \lor b = 0$.\\
Př.:\\
\[\begin{aligned}
x(x+2) = 0\\
x = 0 \lor x = -2\\
K = \{-2;0\}
\end{aligned}\]

Lze též použít \textbf{metodu nulových bodů}\\
Př.:\\
\[\begin{aligned}
4x^2 - 6x < 2x\\
4x^2 - 8x < 0\\
4x(x-2)<0\\
NB = \{0,2\}
\end{aligned}\]

\begin{center}
\begin{tabular}{| c | c | c | c |}
\hline
 & $(-\infty;0)$ & $(0;2)$ & $(2;+\infty)$\\
\hline
$4x$ & - & + & +\\
\hline
$x-2$ & - & - & +\\
\hline
$*$ & + & - & +\\
\hline
\end{tabular}
\end{center}

\[\begin{aligned}
K = (0;2)
\end{aligned}\]

\subsection*{Rovnice s neznámou ve jmenovateli}
Má-li rovnice neznámou ve jmenovateli, je nutné vždy stanovit její definiční obor.

\subsection*{Rovnice v podílovém tvaru}
Rovnice v podílovém tvaru má na jedné straně jediný zlomek s neznámou ve jmenovateli a na druhé straně nulu. Po stanovení definičního oboru řešíme rovnici tak,
že položíme čitatele rovno nule a řešíme jako lineární rovnici nebo převedením na součinový tvar.

\pagebreak

\subsection*{Nerovnice v podílovém tvaru}
\textbf{Nerovnici v podílovém tvaru nesmíme vynásobit společným jmenovatelem, který obsahuje neznámou!}\\
Nerovnici v podílovém tvaru převedeme na podílový tvar a řešíme metodou nulových bodů.\\
Př.:\\

\[
\begin{aligned}
\frac{2x-1}{x+1} \geq 1\\
x \neq -1\\
\frac{2x-1}{x+1} -1 \geq 0\\
\frac{2x-1-(x+1)}{x+1} \geq 0\\
\frac{x-2}{x+1} \geq 0\\
NB = \{-1,2\}
\end{aligned}\]

\begin{center}
\begin{tabular}{| c | c | c | c |}
\hline
 & $(-\infty;-1)$ & $(-1;2 \rangle $ & $ \langle 2;+\infty)$\\
\hline
$x-2$ & - & - & +\\
\hline
$x+1$ & - & + & +\\
\hline
$*$ & + & - & +\\
\hline
\end{tabular}
\end{center}

\[\begin{aligned}
K = (-\infty;-1) \cup \langle 2; +\infty)
\end{aligned}\]

\subsection*{Rovnice s absolutní hodnotou}
\textbf{Absolutní hodnota} z reálného čísla je definována jako\\
\[
   |a| =
    \begin{cases}
        a; & a \geq 0 \\
        -a; & a < 0
    \end{cases}
\]

Také platí :\\
\begin{itemize}

\item $|a|=|-a|$
\item $|a| \geq 0$
\item $|ab|=|a| \cdot |b|$
\item $|\frac{a}{b}| = \frac{|a|}{|b|}$
\end{itemize}

Nejprve určíme argumenty všech absolutních hodnot, ze kterých pak získáme nulové body a intervaly (u intervalu vždy ostrá závorka, kromě NB z jmenovatele). Poté vytvoříme
tabulku jako v tabulkové metodě. Řešíme s upravenými tvary dle definice. \textbf{Ukaždého kořenu musíme ověřit zda leží v intervalu!}\\
Př.:\\

\begin{center}
$|x-3| + 2x = 9$\\
\begin{tabular}{| c | c | c |}
\hline
$x \in $ & $(-\infty; 3 \rangle$ & $\langle 3; +\infty)$\\
\hline
$x-3$ & $-$ & $+$\\
\hline
$|x-3|$ & $(x-3)$ & $x-3$\\
\hline
\end{tabular}
\end{center}

\[
    \begin{array}{ll}
        \text{a)} & \hfill \text{b)}\\
         x \in (-\infty; 3 \rangle & \hfill x \in \langle 3; +\infty)\\
        -(x-3) + 2x = 9 & \hfill x - 3 + 2x = 9\\
        -x+ 3 + 2x = 9 & \hfill 3x - 3 = 9\\
        x = 6 & \hfill x = 4\\
        6 \notin (-\infty; 3 \rangle & \hfill 4 \in \langle 3; +\infty)\\
        K_1 = \{\} & \hfill K_2 = \{4\}\\
    \end{array}
\]\\
Rovnice má tedy jediný kořen $x=4$, tedy $K=\{4\}$.\\

Rovnici ve tvaru $|ax+b|=c$  lze řešit i \textbf{rychlejší metodou} bez stanovení intervalů, neboť platí $ax+b=c \lor ax+b=-c$.\\
I rovnici ve tvaru $|ax+b|=|cx+d|$ lze řešit touto rychlou metodou: $ax+b = cx+d \lor ax+b=-(cx+d)$.\\

\subsection*{Nerovnice s absolutní hodnotou}
Řešíme obdobně jako rovnici s absolutní hodnotou. Z argumentů určíme NB a řešíme nerovnice nahrezené dle definice. Množina řešení je sjednocení množin každého z případů.\\

\begin{center}
\begin{tabular}{| c | c | c |}
\hline
$ x \in $ & $ (-\infty; 1 \rangle $ & $ \langle 1; + \infty) $\\
\hline
$x-1$ & $-$ & $+$\\
\hline
$|x-1|$ & $-(x-1)$ & $x-1$\\
\hline
\end{tabular}
\end{center}

\[
	\begin{array}{ll}
	\text{a)} & \text{b)}\\
	x \in (-\infty, 1 \rangle & \hfill x \in [1;+\infty)\\
	-(x-1) + 3x < 11 & \hfill x - 1 + 3x < 11\\
          -x + 1 + 3x < 11 & \hfill 4x < 12\\
          x < 5 & \hfill x < 3\\
	x \in (-\infty;5) & \hfill x \in (-\infty;3)\\
           K_1 = (-\infty; 1 \rangle \cap (-\infty;5) & \hfill K_2 = (-\infty;3) \cap \langle 1;+\infty)\\
           K_1 = (-\infty;1 \rangle & \hfill K_2 = \langle 1;3)
	\end{array}
\]
\begin{center}
Nerovnice má řešení $K = K_1 \cup K_2 = (-\infty; 3)$.\\
\end{center}

\pagebreak

\section{Soustavy rovnic a nerovnice}
\subsection*{Soustava nerovnic}
\textbf{Soustavu nerovnic} s jednou neznámou řešíme vyřešením každé nerovnice zvlášť. Řešením soustavy je $K_1 \cap K_2$.\\
Př.:\\
\[\begin{aligned}
2x - 5 < 0\\
3x + 2 \geq 0
\end{aligned}\]

\begin{center}
\begin{tabular}{ c c c }
$2x-5 < 0$ & $3x+2 \geq 0$\\
$2x < 5$ & $2x \geq -2$\\
$x < \frac{5}{5}$ & $x \geq -\frac{2}{3}$\\
$K_1 = (-\infty,\frac{5}{2})$ & $K_2 = \langle -\frac{2}{3}, \infty)$
\end{tabular}
\end{center}

\[
\begin{aligned} 
K = K_1 \cap K_2 = \langle - \frac{2}{3}, \frac{5}{2})
\end{aligned}
\]

\textbf{Složenou nerovnici} $ax + b < cx + d < ex + f$ převedeme na soustavu nerovnic:
\[ 
\begin{aligned}
ax + b < cx + d\\
cx + d < ex + f
\end{aligned}
\]

\subsection*{Soustavy rovnic o dvou neznámých}

Soustavu dvou rovnic řešíme buď:\\
\begin{itemize}
\item \textbf{slučovací} (sčítací) - rovnice vhodně vynásobíme a sečteme
\item \textbf{porovnávací} - z každé rovnice vyjádříme tutéž neznámou a porovnáme
\item \textbf{dosazovací} - z jedné rovnice vyjádříme neznámou a dosadíme do druhé 
\end{itemize}

\subsubsection*{Řešení pomocí inverzních matic}
Soustavu m lineárních rovnic o n neznámých ve tvaru\\
\[
\begin{aligned}
a_{11}x_1 + a_{12}x_2 + ... + a_{1n}x_n = b_1\\
a_{21}x_1 + a_{22}x_2 + ... + a_{2n}x_n = b_2\\
...\\
a_{m1}x_1 + a_{m1}x_2 + ... + a_{mn}x_n = b_m\\
\end{aligned}
\]
lze přepsat jako $A \cdot X = B$.\\
Je-li v soustavě rovnic $A \cdot X = B$ matice B rovna nulové matici, mluvíme o homogenní soustavě rovnic. Matici $X$ nazýváme maticí řešení.\\
Maticovou rovnici $A \cdot X = B$ řešíme tak, že vynásobíme obě strany zleva vynásobíme maticí $A^{-1}$\\
\[ \begin{aligned}
\textbf{A} \cdot \textbf{X} = \textbf{B}\\
\textbf{$A^{-1}$} \cdot \textbf{A} \cdot \textbf{X} = \textbf{$A^{-1}$} \cdot \textbf{B}\\
\text{nezapomenout že } A^{-1} \cdot A = E\\
E \cdot \textbf{X} = \textbf{$A^{-1}$} \cdot \textbf{B}\\
\text{zde platí že } E \cdot X = X\\
\textbf{X} = \textbf{$A^{-1}$} \cdot \textbf{B}
\end{aligned} \]\\

Př.:\\
\[
 \begin{aligned}
4x + y = -2\\
3x + y = 5
\end{aligned}\]\\

$A \cdot X = B$, kde $A= \begin{pmatrix} 4 & 1 \\ 3 & 1  \end{pmatrix}$, $B= \begin{pmatrix} -2 \\ 5 \end{pmatrix}$.\\
Určíme inverzní matici $A^{-1}= \begin{pmatrix} 1 & -1 \\ -3 & 4  \end{pmatrix}$ a $X = A^{-1} \cdot B =$\\
$= \begin{pmatrix} 1 & -1 \\ -3 & 4  \end{pmatrix} \cdot \begin{pmatrix} -2 \\ 5 \end{pmatrix} = \begin{pmatrix} -7 \\ 26 \end{pmatrix} \imply x = -7 \land y = 26$\\

Pokud rozšíříme matici soustavy na tvar $(A|B)$, pak platí, že (Frobeniova věta) soustava je řešitelná, když $h(A)=h(A|B)$. Pokud $h(a)$ se rovná počtu neznámých, má soustava jediné řešení.

\subsubsection*{Gaussova eliminační metoda}




\subsection*{Grafické řešení soustavy (ne)rovnic o dvou neznámých}
\subsection*{Soustavy tří rovnic o třech neznámých}
\subsection*{}

\section{Kvadratická rovnice a nerovnice}
\section{Lineární funkce a její vlastnosti}
\section{Kvadratická funkce a její vlastnosti}
\section{Mocninná a lomená funkce a její vlastnosti}
\section{Exponenciální a logaritmická funkce}
\section{Goniometrické funkce}
\section{Množiny bodů dané vlastnosti}
\section{Konstrukce trojúhelníků a čtyřúhelníků}
\section{Shodná zobrazení}
\section{Podobná zobrazení}
\section{Pythagorova a Eukleidovy věty}
\section{Trigonometrie obecného trojúhelníku}
\section{Stereometrie – polohové vlastnosti}
\section{Stereometrie – metrické vlastnosti}
\section{Stereometrie – objem a povrch těles}
\section{Analytická geometrie – body a vektory}
\section{Analytická geometrie – přímka a polorovina v E2}
\section{Analytická geometrie – přímka a rovina v E3}
\section{Analytická geometrie – kuželosečky}
\section{Kombinatorika}
\section{Pravděpodobnost}
\section{Statistika}
\section{Posloupnosti}
\section{Limita posloupnosti a nekonečná geometrická řada}
\section{Limita a derivace funkce}



\end{document}